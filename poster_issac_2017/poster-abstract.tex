\documentclass{article}

\usepackage{bm, bbm, amsmath}
\usepackage{amssymb,stmaryrd}
\usepackage{color}
\usepackage{hyperref}
\usepackage[american]{babel}
\usepackage[utf8]{inputenc}
\usepackage[OT1]{fontenc}

\usepackage{algorithm}
\usepackage{algorithmic}
\renewcommand{\algorithmicrequire}{\textbf{Input:}}
\renewcommand{\algorithmicensure}{\textbf{Output:}}

\def\Q {\ensuremath{\mathbb{Q}}}
\def\Z {\ensuremath{\mathbb{Z}}}
\def\F {\ensuremath{\mathbb{F}}}
\def\Tr {\ensuremath{\mathrm{Tr}}}

\def\M {\ensuremath{\mathsf{M}}}

\newcommand{\todo}[1]{\textcolor{red}{TODO: #1}}

\newtheorem{Def}{Definition}
\newtheorem{Theo}{Theorem}
\newtheorem{Prop}{Proposition}
\newtheorem{Lemma}{Lemma}

\title{Computing isomorphisms and embeddings of finite fields}
\author{Ludovic Brieulle, Luca De Feo, Javad Doliskani,\\ Jean Pierre
  Flori and Éric Schost}


\begin{document}

\maketitle

Let $q$ be a prime power and let $\F_q$ be a field with $q$
elements. Let $f$ and $g$ be irreducible polynomials in $\F_q[X]$,
with $\deg f$ dividing $\deg g$. Define $k=\F_q[X]/f$ and
$K=\F_q[X]/g$, then there is an embedding $\varphi:k\hookrightarrow
K$, unique up to \mbox{$\F_q$-auto}morphisms of $k$. Our goal is to
describe algorithms to efficiently represent and evaluate one such
embedding.

The special case where $\deg f = \deg g$ is also called the
\emph{isomorphism problem}. These algorithms have important
applications in the representation of finite fields and their
extensions. Consequently, they are widely applied in cryptography and
coding theory. In particular, they are useful for representing the
algebraic closure $\bar{\F}_q$, as described
in~\cite{bosma+cannon+steel97}. They are fundamental building blocks
of most computer algebra systems, including Magma\cite{MAGMA},
Pari\cite{Pari}, Sage\cite{Sage}, ...

All the algorithms we are aware of, split the embedding problem in two
sub-problems:
\begin{enumerate}
\item Determine elements $\alpha\in k$ and $\beta\in K$ such that
  $k=\F_q[\alpha]$, $K=\F_q[\beta]$, and such that there exists a
  uniquely defined embedding $\varphi$ mapping
  $\alpha\mapsto\beta$. We refer to this problem as the
  \emph{Embedding description}.
\item Given elements $\alpha$ and $\beta$ as above, given $\gamma\in
  k$ and $\delta\in K$, solve the following problems:
  \begin{itemize}
  \item Compute $\varphi(\gamma)$ in $K$.
  \item Test if $\delta\in\varphi(k)$.
  \item Supposing $\delta\in\varphi(k)$, compute $\varphi^{-1}(\delta)$ in $K$.
  \end{itemize}
  We refer collectively to these problems as the \emph{Embedding
    evaluation}.
\end{enumerate}

Arguably, the hardest problem among the two is the embedding
description. %
The first to get interested in this problem was H.~Lenstra: in his
seminal paper~\cite{LenstraJr91} he shows that it can be solved in
deterministic polynomial time, by using a representation for finite
fields that he calls \emph{explicit data}.\footnote{Technically,
  Lenstra only proved his theorem in the case where $k$ and $K$ are
  isomorphic; however, the generalization to the embedding problem
  poses no difficulties.} %
The first computer algebra system system to implement a general
embedding algorithm was Magma~\cite{MAGMA}. %
As detailed by its developers~\cite{bosma+cannon+steel97}, it used a
much simpler approach than Lenstra's algorithm, entirely based on
polynomial factorization and linear algebra. %
Lenstra's algorithm was later revived by
Allombert~\cite{Allombert02,Allombert02-rev} who modified some key
steps in order to make it practical; his implementation has since been
part of the PARI/GP system~\cite{Pari}.

Meanwhile, a distinct family of algorithms for the embedding problem
was started by Pinch~\cite{Pinch}, and later improved by
Rains~\cite{rains2008}. %
These algorithms, based on principles radically different from
Lenstra's, are intrinsically probabilistic. %
Although their worst-case complexity is no better than that of
Allombert's algorithm, they are potentially much more efficient on a
large set of parameters. %
This potential was understood by Magma's developers, who implemented
Rains' algorithm in Magma v2.14.%
\footnote{As a matter of fact, Rains' algorithm was never published;
  the only publicly available source for it is in Magma's source code
  (file \texttt{package/Ring/FldFin/embed.m}, since v2.14).}

With the exception of Lenstra's work, the aforementioned papers were
mostly concerned with the practical aspects of the embedding
problem. %
While it was generally understood that computing embeddings is an
easier problem than general polynomial factoring, no results on its
complexity more precise than Lenstra's had appeared until recently. %
Recently, Narayanan published a novel generalization of Allombert's
algorithm~\cite{narayanan2016fast}, based on elliptic curve
computations, and showed that its (randomized) complexity is at most
quadratic.

\paragraph{Our contribution}
This poster reviews all known algorithms for the embedding
problem. We analyze in detail the cost of existing algorithms and
introduce several new variants. %
We divide the algorithms in two families:
\begin{itemize}
\item \emph{Kummer-type} algorithms, based (more or less loosely) on
  Lenstra's work, exploit the properties of $n$-th roots of
  unity, with $n=\deg f$;
\item \emph{Group-based} algorithms, base on Rains' algorithm, which
  exploit the properties of some algebraic group $G/\F_q$.
\end{itemize}

We pay a particular attention to Allombert's algorithm:
to our knowledge, this is the first detailed and complete complexity
analysis of this algorithm and its variants. %
Thanks to our work on asymptotic complexity, we were able to devise
improvements to the original variants of Allombert that largely
outperform them both in theory and practice. %

We also present a generalization of Rains' algorithm using elliptic
curves. %
The possibility of this algorithm was hinted at by Rains, but never
fully developed; we show that it is indeed possible to use
\emph{elliptic periods} to solve the embedding description problem,
and that the resulting algorithm behaves well both in theory and in
practice. %
While working out the correctness proof of the elliptic variant of
Rains' algorithm, we encounter an unexpected difficulty: whereas roots
of unity enjoy Galois properties that guarantee the success of Rains'
original algorithm, points of elliptic curves fail to provide the
same. %
Heuristically, the failure probability of the elliptic variant is
extremely small, however we are not able to prove it formally. %
Our experimental searches even seem to suggest that the failure
probability might be, surprisingly, zero. %
We state this as a conjecture on elliptic periods; our give some
supporting evidence.

Finally, we review to the known strategies for embedding
evaluation. Besides the obvious approach based on linear algebra, we
present techniques based on modular composition and duality.

We complement our findings with experimental results, by comparing our
implementations of all the algorithms. Our source code is made
available through the Git repository
\url{https://github.com/defeo/ffisom} for replication and further
scrutiny.

\bibliographystyle{plain}
\bibliography{../refs}

\end{document}


% Local Variables:
% ispell-local-dictionary:"american"
% End:

%  LocalWords:  isomorphism
