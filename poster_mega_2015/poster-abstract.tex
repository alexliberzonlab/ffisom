\documentclass{article}

\usepackage{bm, bbm, amsmath}
\usepackage{amssymb,stmaryrd}
\usepackage{color}
\usepackage{hyperref}
\usepackage[american]{babel}
\usepackage[utf8]{inputenc}
\usepackage[OT1]{fontenc}

\usepackage{algorithm}
\usepackage{algorithmic}
\renewcommand{\algorithmicrequire}{\textbf{Input:}}
\renewcommand{\algorithmicensure}{\textbf{Output:}}

\def\Q {\ensuremath{\mathbb{Q}}}
\def\Z {\ensuremath{\mathbb{Z}}}
\def\F {\ensuremath{\mathbb{F}}}
\def\Tr {\ensuremath{\mathrm{Tr}}}

\def\M {\ensuremath{\mathsf{M}}}

\newcommand{\todo}[1]{\textcolor{red}{TODO: #1}}

\newtheorem{Def}{Definition}
\newtheorem{Theo}{Theorem}
\newtheorem{Prop}{Proposition}
\newtheorem{Lemma}{Lemma}

\title{Computing isomorphisms and embeddings of finite fields}
\author{Ludovic Brieulle, Luca De Feo, Javad Doliskani,\\ Jean Pierre
  Flori and Éric Schost}


\begin{document}

\maketitle

Let $q$ be a prime power and let $\F_q$ be a field with $q$
elements. Let $f$ and $g$ be irreducible polynomials in $\F_q[X]$,
with $\deg f$ dividing $\deg g$. Define $k=\F_q[X]/f$ and
$K=\F_q[X]/g$, then there is an embedding $\varphi:k\hookrightarrow
K$, unique up to \mbox{$\F_q$-auto}morphisms of $k$. Our goal is to
describe algorithms to efficiently represent and evaluate one such
embedding.

The special case where $\deg f = \deg g$ is also called the
\emph{isomorphism problem}. These algorithms have important
applications in the representation of finite fields and their
extensions. Consequently, they are widely applied in cryptography and
coding theory. In particular, they are useful for representing the
algebraic closure $\bar{\F}_q$, as described
in~\cite{bosma+cannon+steel97}. They are fundamental building blocks
of most computer algebra systems, including Magma\cite{MAGMA},
Pari\cite{Pari}, Sage\cite{Sage}, ...

All the algorithms we are aware of, split the embedding problem in two
sub-problems:
\begin{enumerate}
\item Determine elements $\alpha\in k$ and $\beta\in K$ such that
  $k=\F_q[\alpha]$, $K=\F_q[\beta]$, and such that there exists a
  uniquely defined embedding $\varphi$ mapping
  $\alpha\mapsto\beta$. We refer to this problem as the
  \emph{Embedding description}.
\item Given elements $\alpha$ and $\beta$ as above, given $\gamma\in
  k$ and $\delta\in K$, solve the following problems:
  \begin{itemize}
  \item Compute $\varphi(\gamma)$ in $K$.
  \item Test if $\delta\in\varphi(k)$.
  \item Supposing $\delta\in\varphi(k)$, compute $\varphi^{-1}(\delta)$ in $K$.
  \end{itemize}
  We refer collectively to these problems as the \emph{Embedding
    evaluation}.
\end{enumerate}

Arguably, the hardest problem among the two is the embedding
description. To tackle it, we review two families of algorithms:
\begin{itemize}
\item Kummer-type algorithms, which exploit the properties of $n$-th
  roots of unity, with $n=\deg f$.
\item Group-based algorithms, which exploit the properties of some
  algebraic group $G/\F_q$.
\end{itemize}

In the first category fall Lenstra's deterministic isomorphism
algorithm\cite{LenstraJr91}, and Allombert's
variant\cite{Allombert02}. We present these two algorithms, and give
an asymptotic improvement to the latter.

In the second category fall Pinch's elliptic curve
algorithm~\cite{Pinch}, and Rains' cyclotomic
algorithm~\cite{rains2008}. We review those algorithms and introduce
the elliptic curve version of Rains' algorithm. Although Rains had
predicted that an elliptic curve variant could not improve over the
plain cyclotomic algorithm, we show that this is not true in general,
and that for many parameter families the elliptic algorithm is indeed
faster.

Neither the Kummer-type, nor the group-based algorithms are
asymptotically optimal. However, they succeed in being close to
optimal for different parameter families, and, put together, cover in
practice most of the range one is typically interested in. Hence, a
complete implementation shall include both families, and carefully
choose among them based on the input parameters. We highlight this by
implementing the fastest variants of each of the algorithms presented,
and comparing them on a wide array of benchmarks.

Next, we go to the known strategies for embedding evaluation. Besides
the obvious approach based on linear algebra, we present techniques
based on modular composition, and techniques based on normal
bases. These advanced techniques are folklore, but they are seldom
presented in a comprehensive way.

For the embedding evaluation problem too, we provide implementations
and benchmarks comparing the various approaches.


\bibliographystyle{plain}
\bibliography{../defeo}

\end{document}


% Local Variables:
% ispell-local-dictionary:"american"
% End:

%  LocalWords:  isomorphism
